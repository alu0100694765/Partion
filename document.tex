%%%%%%%%%%%%%%%%%%%%%%%%%%%%%%%%%%%%%%%%%
%
% Paritition Presentation
% LaTeX 
% Version 1.0 (08/12/14)
%
%%%%%%%%%%%%%%%%%%%%%%%%%%%%%%%%%%%%%%%%%

%----------------------------------------------------------------------------------------
%	PACKAGES AND THEMES
%----------------------------------------------------------------------------------------

\documentclass{beamer}

\mode<presentation> {
\usetheme{Madrid}
}

\usepackage{graphicx} % Permite introducir imagenes
\usepackage{booktabs} % Allows the use of \toprule, \midrule and \bottomrule in tables
\usepackage{amsfonts}  % Simbolos
\usepackage{amssymb}  % Simbolos
\usepackage{amsmath} % matematicos

%----------------------------------------------------------------------------------------
%	TITLE PAGE
%----------------------------------------------------------------------------------------

% Titulo de la presentacion
\title[Short title]{ Partition }

% Autores
\author{Sawan J. Kapai Harpalani \and
	Adri\'an Gonz\'alez Mart\'in \and
	Sara Mart\'in Molina \and
	Enrique Tejera Gonz\'alez
}

% Institucion a la que pertenecen los colaboradores
\institute[ULL] 
{
Universidad de La Laguna \\ 
\medskip
}

% Fecha de hoy
\date{\today} 

% Comienzo del documento
\begin{document}

\begin{frame}

% Imprime el titulo
\titlepage 

\end{frame}


%----------------------------------------------------------------------------------------
%	DIAPOSITIVAS DE LA PRESENTACION
%----------------------------------------------------------------------------------------

\begin{frame}
\frametitle{Partition}
% Comenzamos una enumeracion
\begin{itemize}
  \item Teorema: Partition es NP-Completo.
  \item Instancia:  A  a $\in$ A S(a) $\in$ \(\mathbb{Z}\)$\textsuperscript{+}$.
  \item Prueba: Es f\'acil ver que partition $\in$ NP, puesto que es un algoritmo
  no determinista necesita s\'olo encontrar un subconjunto A' de A y comprobar el
  tiempo polinomial que suma los tama\~{n}os de los elementos de A' es igual a la
  suma de los elementos de A-A'.
  
\end{itemize}  
\end{frame}

%------------------------------------------------

\begin{frame}
\frametitle{Transformaci\'on 3DM a Partition}
\begin{itemize}
\item Se fijan los conjuntos W, X, Y con tamaño q y M que ser\'a una instancia
arbitraria del 3DM (M $\subseteq$ $W \times X \times Y$).
	$$W = w_{1}, w_{2}, w_{3}, \cdot\ldots\cdot w_{q}$$
	$$X = x_{1}, x_{2}, x_{3}, \cdot\ldots\cdot x_{q}$$
	$$Y = y_{1}, y_{2}, y_{3}, \cdot\ldots\cdot y_{q}$$
	$$M = m_{1}, m_{2}, m_{3}, \cdot\ldots\cdot m_{k}$$ 
	$$ k = |M|$$
	
\item Se debe construir un conjunto A, donde cada elemento tiene tama\~{n}o tal que S(a) $\in$ \(\mathbb{Z}\)$\textsuperscript{+}$ y ese A debe contener un subconjunto A' tal que:
	$$\sum\limits_{a \in A'} S(a) = \sum\limits_{a \in A - A'} S(a) \iff matching(M)$$
\item El conjunto A contendr\'a k + 2 elementos.
\end{itemize}
\end{frame}


%------------------------------------------------

\begin{frame}
\frametitle{Construcci\'on}
Se construye en dos pasos:
\begin{enumerate}
\item Primer paso:
\begin{itemize}
\item Los primeros k elementos de A est\'an asociados con las k tripletas de M
	  $$ a_{i} \Rightarrow m_{i}, 1 \leq i \leq k $$
\item El tama\~{n}o de cada elementos se obtiene de su representaci\'on binaria.
	  $$ |w| = |x| = |y| = q \Rightarrow p $$
      $$p = [\log_2(k + 1)]$$
\end{itemize}

\end{enumerate}

\end{frame}

%------------------------------------------------

\begin{frame}
\frametitle{Construcci\'on}
\begin{enumerate}
\item Primer paso:
\begin{itemize}
\item La representaci\'on para el tama\~{n}o del elemento $a_{i}$ depende de la tripleta:
	$$m_{i} = (w_{f(i)}, x_{g(i)}, y_{h(i)}) \in matching(M)$$
\item Otra forma de obtener el tama\~{n}o de $a_{i}$:
	$$S(a_{i}) = 2\textsuperscript{p(3q - f(i))} + 2\textsuperscript{p(2q - g(i))} + 2\textsuperscript{p(q - h(i))}$$
\end{itemize}

\end{enumerate}

\end{frame}

%------------------------------------------------

\begin{frame}
\frametitle{Construcci\'on}
\begin{enumerate}
\item Primer paso:
\begin{itemize}
\item Si se fija:
	$$B = \sum\limits_{j=0}^{3q -1} 2\textsuperscript{pj}$$
\item Entonces:
	$$A'\subseteq {a_{i}: 1 \leq i \leq k}$$
    $$\sum\limits_{a\in A'} = B \iff M' = {m_{i}: a_{i} \in A' \Rightarrow matching(M)}$$
\end{itemize}

\end{enumerate}

\end{frame}


%------------------------------------------------

\begin{frame}
\frametitle{Construcci\'on}
\begin{enumerate}
\setcounter{enumi}{1}
\item Segundo paso:
\begin{itemize}
\item Se especifican los dos \'ultimos elementos de A ($b_{1}$ y $b_{2}$) cuyos tama\~{n}os son:
	$$S(b_{1}) = 2(\sum\limits_{i=1}^{k}S(a_{i})) - B = B$$
    $$S(b_{2}) = 2(\sum\limits_{i=1}^{k}S(a_{i})) + B = 2B$$
\item Ambos pueden ser especificados en binario con m\'as de (3pq + 1) bits.
\end{itemize}

\end{enumerate}

\end{frame}

%------------------------------------------------

\begin{frame}
\frametitle{Construcci\'on}
\begin{enumerate}
\setcounter{enumi}{1}
\item Segundo paso:
\begin{itemize}
\item Suponiendo que se tiene un conjunto $A' \subseteq A$ se cumple:
	$$\sum\limits_{a \in A'} S(a) = \sum\limits_{a \in A - A'} S(a)$$
\item Por lo que las sumas de ambos ser\'a:
	$$ 2\sum\limits_{i=1}^{k}S(a_{i})$$
 \item A o A - A1 contendr\'a $b_{1}$ pero no $b_{2}$.
\end{itemize}

\end{enumerate}

\end{frame}

%------------------------------------------------

\begin{frame}
\frametitle{Construcci\'on}
\begin{enumerate}
\setcounter{enumi}{1}
\item Segundo paso:
\begin{itemize}
\item El resto de elementos formar\'an un conjunto 
	$$a_{i}: 1 \leq i \leq k$$
\item La suma de los tama\~{n}os suman B por lo que es un matching de M' en M.
\item A la inversa:
	$$M' \subseteq M \Rightarrow matching(M) \Rightarrow {b_{1} \bigcup {a_{1}}: m_{i} \in M'} $$
\end{itemize}

\end{enumerate}

\end{frame}

%------------------------------------------------

\begin{frame}
\Huge{\centerline{FIN}}
\end{frame}

%----------------------------------------------------------------------------------------

\end{document}