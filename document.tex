%%%%%%%%%%%%%%%%%%%%%%%%%%%%%%%%%%%%%%%%%
%
% Paritition Presentation
% LaTeX 
% Version 1.0 (08/12/14)
%
%%%%%%%%%%%%%%%%%%%%%%%%%%%%%%%%%%%%%%%%%

%----------------------------------------------------------------------------------------
%	PACKAGES AND THEMES
%----------------------------------------------------------------------------------------

\documentclass{beamer}

\mode<presentation> {
\usetheme{Madrid}
}

\usepackage{graphicx} % Permite introducir imagenes
\usepackage{booktabs} % Allows the use of \toprule, \midrule and \bottomrule in tables
\usepackage{amsfonts}  % Simbolos
\usepackage{amssymb}  % Simbolos

%----------------------------------------------------------------------------------------
%	TITLE PAGE
%----------------------------------------------------------------------------------------

% Titulo de la presentacion
\title[Short title]{ Partition }

% Autores
\author{Sawan J. Kapai Harpalani \and
	Adri\'an Gonz\'alez Mart\'in \and
	Sara Mart\'in Molina \and
	Enrique Tejera Gonz\'alez
}

% Institucion a la que pertenecen los colaboradores
\institute[ULL] 
{
Universidad de La Laguna \\ 
\medskip
}

% Fecha de hoy
\date{\today} 

% Comienzo del documento
\begin{document}

\begin{frame}

% Imprime el titulo
\titlepage 

\end{frame}

\begin{frame}

% Tabla de contenidos, quitar si no se desea 
\frametitle{Overview} 

% Throughout your presentation, if you choose to use \section{} and
% \subsection{} commands, these will automatically be printed on this slide as an overview of your presentation
\tableofcontents 
\end{frame}

%----------------------------------------------------------------------------------------
%	DIAPOSITIVAS DE LA PRESENTACION
%----------------------------------------------------------------------------------------

%------------------------------------------------
% Las secciones sirven para organizar las tablas de contenidos
% todos los bloques seran visibles en la tabla de contenidos
\section{Partition} 
%------------------------------------------------

% Ejemplo de subseccion
\subsection{Subsection Example} 

\begin{frame}
\frametitle{Partition}
% Comenzamos una enumeracion
\begin{itemize}
  \item Teorema: Partition es NP-Completo.
  \item Instancia:  A  a $\in$ A S(a) $\in$ \(\mathbb{Z}\)$\textsuperscript{+}$.
  \item Prueba: Es f\'acil ver que partition $\in$ NP, puesto que es un algoritmo
  no determinista necesita s\'olo encontrar un subconjunto A' de A y comprobar el
  tiempo polinomial que suma los tama\~{n}os de los elementos de A' es igual a la
  suma de los elementos de A-A'.
  
\end{itemize}  
\end{frame}

%------------------------------------------------

\begin{frame}
\frametitle{Transformaci\'on 3DM a Partition}
\begin{itemize}
\item Se fijan los conjuntos W, X, Y con tamaño q y M que ser\'a una instancia
arbitraria del 3DM (M $\subseteq$ $W \times X \times Y$).
	$$W = w_{1}, w_{2}, w_{3}, \cdot\ldots\cdot w_{q}$$
	$$X = x_{1}, x_{2}, x_{3}, \cdot\ldots\cdot x_{q}$$
	$$Y = y_{1}, y_{2}, y_{3}, \cdot\ldots\cdot y_{q}$$
	$$M = m_{1}, m_{2}, m_{3}, \cdot\ldots\cdot m_{k}$$ 
	$$ k = |M|$$
	
\item Se debe construir un conjunto A, donde cada elemento tiene tama\~{n}o tal que S(a) $\in$ \(\mathbb{Z}\)$\textsuperscript{+}$ y ese A debe contener un subconjunto A' tal que:
	$$\sum\limits_{a \in A'} S(a) = \sum\limits_{a \in A - A'} S(a) \iff matching(M)$$
\item El conjunto A contendr\'a k + 2 elementos.
\end{itemize}
\end{frame}


%------------------------------------------------

\begin{frame}
\Huge{\centerline{FIN}}
\end{frame}

%----------------------------------------------------------------------------------------

\end{document}